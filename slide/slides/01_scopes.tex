\section{Mục tiêu, giới hạn và đối tượng của nghiên cứu}\label{sec:intro}
\frame{\tableofcontents[currentsection]}
\begin{frame}{Mục tiêu, giới hạn và đối tượng của nghiên cứu}
\begin{enumerate}
    \item <1-> \textit{Mục tiêu}: Xây dựng mô hình có khả năng tạo sinh hình ảnh khuôn mặt người một cách tự nhiên, chính xác.
    \item <2-> \textit{Giới hạn}: Tạo sinh hình ảnh trong vùng mặt người. Dữ liệu mẫu được cung cấp ban đầu phải là hình ảnh rõ ràng của khuôn mặt người. Đoạn âm thanh cũng phải là đoạn tiếng nói rõ ràng, cùng loại với ngôn ngữ được dùng để huấn luyện mạng.
    \item <3-> \textit{Đối tượng}: Các phương pháp mô hình hóa bài toán, học máy, học sâu, mạng GANs và các phương pháp tạo sinh dữ liệu từ mạng GANs, các phương pháp kết hợp đặc trưng hình ảnh, âm thanh để tạo sinh dữ liệu mới.
\end{enumerate}
\end{frame}
