\section{\texorpdfstring{Mục tiêu, giới hạn và đối tượng nghiên cứu}{Target, limitation}}
\subsection{\texorpdfstring{Mục tiêu}{Target}}
\label{target_label}
Mục tiêu của Luận văn Tốt nghiệp là nghiên cứu các đề tài có liên quan bằng cách khảo sát, kiểm định và thử nghiệm các nghiên cứu mới nhất hiện có, qua đó tiến hành các cải tiến, thay đổi và thử nghiệm để đưa ra các kết quả tạo sinh khuôn mặt tốt hơn, tự nhiên hơn, chính xác hơn. Hình ảnh được tạo ra phải sắc nét, ít nhiễu, chân thực và tương đồng về mặt nhận dạng, cấu trúc với hình ảnh người mẫu. Đồng thời, khẩu hình miệng của hình ảnh được tạo ra phải khớp với tiếng nói, phù hợp với cách phát âm từ ngữ. Bên cạnh đó, video được tạo ra phải có tính liền lạc, ổn định, không bị hiện tượng nhảy hình. Mục tiêu được đặt ra nhằm cải thiện các mô hình hiện có, tăng tính ứng dụng của việc tạo sinh mặt người vào thực tiễn cuộc sống.

\subsection{\texorpdfstring{Giới hạn}{Limitation}}
Phạm vi nghiên cứu của Luận văn là tạo sinh ảnh giới hạn trong vùng mặt của người, dữ liệu mẫu được cung cấp ban đầu phải là ảnh rõ ràng của khuôn mặt người, đoạn âm thanh được cung cấp cũng phải là âm thanh rõ ràng của tiếng nói cùng loại với ngôn ngữ được dùng để huấn luyện mạng.

\subsection{\texorpdfstring{Đối tượng nghiên cứu}{Research}}
Đối tượng nghiên cứu của Luận văn là các cách tiếp cận, các phương pháp mô hình hóa bài toán, các mạng học máy, học sâu, mạng GANs và các phương pháp tạo sinh dữ liệu từ mạng GANs, các cấu trúc Residual Encoder-Decoder, bên cạnh đó là các phương pháp kết hợp đặc trưng hình ảnh, âm thanh có xem xét đến thứ tự thời gian để tạo sinh hình ảnh mới.

\subsection{\texorpdfstring{Kết quả dự kiến}{Result}}
Luận văn sẽ xây dựng được mô hình tính toán mới để có thể tạo sinh hình ảnh mặt người hiệu quả, thỏa mãn được các tiêu chí đã nêu trong phần \ref{target_label}. Đồng thời, luận văn cũng sẽ cung cấp được các đánh giá và so sánh khách quan bằng số liệu thực tế.
