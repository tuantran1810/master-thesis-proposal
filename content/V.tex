\section{\texorpdfstring{Tổng quan các công trình nghiên cứu liên quan}{Content}}
Việc tạo sinh dữ liệu mới trong thời gian gần đây đã phát triển mạnh mẽ với sự ra đời của kiến trúc mạng GANs. Hệ thống mạng GANs bao gồm các mạng học máy, học sâu nhỏ hơn, chia thành hai thành phân là mạng tạo sinh dữ liệu (mạng G) và mạng phân biệt dữ liệu (mạng D). Mạng G đóng vai trò như một Variational Autoencoder \cite{vae_base},  có chức năng học và xấp xỉ được phân phối xác suất của dữ liệu gốc, từ đó tạo sinh ra dữ liệu mới giữ được đặc trưng và tương đồng với dữ liệu gốc. Mạng D có chức năng phân biệt giữa dữ liệu được tạo sinh bởi mạng G và tập dữ liệu huấn luyện. Trong quá trình huấn luyện, dựa trên hàm mất mát của D, các trọng số của cả hai mạng G và D đều được cập nhật trong quá trình lan truyền ngược, từ đó giúp hai mạng này tăng độ chính xác. Với mạng G, qua quá trình huấn luyện, mạng sẽ có khả năng tạo sinh ra được dữ liệu ngày càng chân thực hơn, khó phân biệt hơn. Trong khi đó, mạng D cũng ngày càng có chức năng phân biệt tốt hơn, chuẩn xác hơn. Đến một lúc nào đó, độ chính xác của hai mạng sẽ đạt đến mức cân bằng, lúc này hai mạng đã hội tụ và không thể được cải thiện hơn với kiến trúc mạng và tập dữ liệu huấn luyện hiện tại, nên ta sẽ dừng quá trình huấn luyện tại đây.


